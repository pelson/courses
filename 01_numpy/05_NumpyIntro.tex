
% Default to the notebook output style

    


% Inherit from the specified cell style.




    
\documentclass{article}

    
    
    \usepackage{graphicx} % Used to insert images
    \usepackage{adjustbox} % Used to constrain images to a maximum size 
    \usepackage{color} % Allow colors to be defined
    \usepackage{enumerate} % Needed for markdown enumerations to work
    \usepackage{geometry} % Used to adjust the document margins
    \usepackage{amsmath} % Equations
    \usepackage{amssymb} % Equations
    \usepackage[utf8]{inputenc} % Allow utf-8 characters in the tex document
    \usepackage[mathletters]{ucs} % Extended unicode (utf-8) support
    \usepackage{fancyvrb} % verbatim replacement that allows latex
    \usepackage{grffile} % extends the file name processing of package graphics 
                         % to support a larger range 
    % The hyperref package gives us a pdf with properly built
    % internal navigation ('pdf bookmarks' for the table of contents,
    % internal cross-reference links, web links for URLs, etc.)
    \usepackage{hyperref}
    \usepackage{longtable} % longtable support required by pandoc >1.10
    

    
    
    \definecolor{orange}{cmyk}{0,0.4,0.8,0.2}
    \definecolor{darkorange}{rgb}{.71,0.21,0.01}
    \definecolor{darkgreen}{rgb}{.12,.54,.11}
    \definecolor{myteal}{rgb}{.26, .44, .56}
    \definecolor{gray}{gray}{0.45}
    \definecolor{lightgray}{gray}{.95}
    \definecolor{mediumgray}{gray}{.8}
    \definecolor{inputbackground}{rgb}{.95, .95, .85}
    \definecolor{outputbackground}{rgb}{.95, .95, .95}
    \definecolor{traceback}{rgb}{1, .95, .95}
    % new ansi colors
    \definecolor{brown}{rgb}{0.54,0.27,0.07}
    \definecolor{purple}{rgb}{0.5,0.0,0.5}
    \definecolor{darkgray}{gray}{0.25}
    \definecolor{lightred}{rgb}{1.0,0.39,0.28}
    \definecolor{lightgreen}{rgb}{0.48,0.99,0.0}
    \definecolor{lightblue}{rgb}{0.53,0.81,0.92}
    \definecolor{lightpurple}{rgb}{0.87,0.63,0.87}
    \definecolor{lightcyan}{rgb}{0.5,1.0,0.83}
    % Define a nice break command that doesn't care if a line doesn't already
    % exist.
    \def\br{\hspace*{\fill} \\* }
    % Math Jax compatability definitions
    \def\gt{>}
    \def\lt{<}
    % Document parameters
    \title{05\_NumpyIntro}
    
    
    

    % Pygments definitions
    
\makeatletter
\def\PY@reset{\let\PY@it=\relax \let\PY@bf=\relax%
    \let\PY@ul=\relax \let\PY@tc=\relax%
    \let\PY@bc=\relax \let\PY@ff=\relax}
\def\PY@tok#1{\csname PY@tok@#1\endcsname}
\def\PY@toks#1+{\ifx\relax#1\empty\else%
    \PY@tok{#1}\expandafter\PY@toks\fi}
\def\PY@do#1{\PY@bc{\PY@tc{\PY@ul{%
    \PY@it{\PY@bf{\PY@ff{#1}}}}}}}
\def\PY#1#2{\PY@reset\PY@toks#1+\relax+\PY@do{#2}}

\expandafter\def\csname PY@tok@gd\endcsname{\def\PY@tc##1{\textcolor[rgb]{0.63,0.00,0.00}{##1}}}
\expandafter\def\csname PY@tok@gu\endcsname{\let\PY@bf=\textbf\def\PY@tc##1{\textcolor[rgb]{0.50,0.00,0.50}{##1}}}
\expandafter\def\csname PY@tok@gt\endcsname{\def\PY@tc##1{\textcolor[rgb]{0.00,0.27,0.87}{##1}}}
\expandafter\def\csname PY@tok@gs\endcsname{\let\PY@bf=\textbf}
\expandafter\def\csname PY@tok@gr\endcsname{\def\PY@tc##1{\textcolor[rgb]{1.00,0.00,0.00}{##1}}}
\expandafter\def\csname PY@tok@cm\endcsname{\let\PY@it=\textit\def\PY@tc##1{\textcolor[rgb]{0.25,0.50,0.50}{##1}}}
\expandafter\def\csname PY@tok@vg\endcsname{\def\PY@tc##1{\textcolor[rgb]{0.10,0.09,0.49}{##1}}}
\expandafter\def\csname PY@tok@m\endcsname{\def\PY@tc##1{\textcolor[rgb]{0.40,0.40,0.40}{##1}}}
\expandafter\def\csname PY@tok@mh\endcsname{\def\PY@tc##1{\textcolor[rgb]{0.40,0.40,0.40}{##1}}}
\expandafter\def\csname PY@tok@go\endcsname{\def\PY@tc##1{\textcolor[rgb]{0.53,0.53,0.53}{##1}}}
\expandafter\def\csname PY@tok@ge\endcsname{\let\PY@it=\textit}
\expandafter\def\csname PY@tok@vc\endcsname{\def\PY@tc##1{\textcolor[rgb]{0.10,0.09,0.49}{##1}}}
\expandafter\def\csname PY@tok@il\endcsname{\def\PY@tc##1{\textcolor[rgb]{0.40,0.40,0.40}{##1}}}
\expandafter\def\csname PY@tok@cs\endcsname{\let\PY@it=\textit\def\PY@tc##1{\textcolor[rgb]{0.25,0.50,0.50}{##1}}}
\expandafter\def\csname PY@tok@cp\endcsname{\def\PY@tc##1{\textcolor[rgb]{0.74,0.48,0.00}{##1}}}
\expandafter\def\csname PY@tok@gi\endcsname{\def\PY@tc##1{\textcolor[rgb]{0.00,0.63,0.00}{##1}}}
\expandafter\def\csname PY@tok@gh\endcsname{\let\PY@bf=\textbf\def\PY@tc##1{\textcolor[rgb]{0.00,0.00,0.50}{##1}}}
\expandafter\def\csname PY@tok@ni\endcsname{\let\PY@bf=\textbf\def\PY@tc##1{\textcolor[rgb]{0.60,0.60,0.60}{##1}}}
\expandafter\def\csname PY@tok@nl\endcsname{\def\PY@tc##1{\textcolor[rgb]{0.63,0.63,0.00}{##1}}}
\expandafter\def\csname PY@tok@nn\endcsname{\let\PY@bf=\textbf\def\PY@tc##1{\textcolor[rgb]{0.00,0.00,1.00}{##1}}}
\expandafter\def\csname PY@tok@no\endcsname{\def\PY@tc##1{\textcolor[rgb]{0.53,0.00,0.00}{##1}}}
\expandafter\def\csname PY@tok@na\endcsname{\def\PY@tc##1{\textcolor[rgb]{0.49,0.56,0.16}{##1}}}
\expandafter\def\csname PY@tok@nb\endcsname{\def\PY@tc##1{\textcolor[rgb]{0.00,0.50,0.00}{##1}}}
\expandafter\def\csname PY@tok@nc\endcsname{\let\PY@bf=\textbf\def\PY@tc##1{\textcolor[rgb]{0.00,0.00,1.00}{##1}}}
\expandafter\def\csname PY@tok@nd\endcsname{\def\PY@tc##1{\textcolor[rgb]{0.67,0.13,1.00}{##1}}}
\expandafter\def\csname PY@tok@ne\endcsname{\let\PY@bf=\textbf\def\PY@tc##1{\textcolor[rgb]{0.82,0.25,0.23}{##1}}}
\expandafter\def\csname PY@tok@nf\endcsname{\def\PY@tc##1{\textcolor[rgb]{0.00,0.00,1.00}{##1}}}
\expandafter\def\csname PY@tok@si\endcsname{\let\PY@bf=\textbf\def\PY@tc##1{\textcolor[rgb]{0.73,0.40,0.53}{##1}}}
\expandafter\def\csname PY@tok@s2\endcsname{\def\PY@tc##1{\textcolor[rgb]{0.73,0.13,0.13}{##1}}}
\expandafter\def\csname PY@tok@vi\endcsname{\def\PY@tc##1{\textcolor[rgb]{0.10,0.09,0.49}{##1}}}
\expandafter\def\csname PY@tok@nt\endcsname{\let\PY@bf=\textbf\def\PY@tc##1{\textcolor[rgb]{0.00,0.50,0.00}{##1}}}
\expandafter\def\csname PY@tok@nv\endcsname{\def\PY@tc##1{\textcolor[rgb]{0.10,0.09,0.49}{##1}}}
\expandafter\def\csname PY@tok@s1\endcsname{\def\PY@tc##1{\textcolor[rgb]{0.73,0.13,0.13}{##1}}}
\expandafter\def\csname PY@tok@sh\endcsname{\def\PY@tc##1{\textcolor[rgb]{0.73,0.13,0.13}{##1}}}
\expandafter\def\csname PY@tok@sc\endcsname{\def\PY@tc##1{\textcolor[rgb]{0.73,0.13,0.13}{##1}}}
\expandafter\def\csname PY@tok@sx\endcsname{\def\PY@tc##1{\textcolor[rgb]{0.00,0.50,0.00}{##1}}}
\expandafter\def\csname PY@tok@bp\endcsname{\def\PY@tc##1{\textcolor[rgb]{0.00,0.50,0.00}{##1}}}
\expandafter\def\csname PY@tok@c1\endcsname{\let\PY@it=\textit\def\PY@tc##1{\textcolor[rgb]{0.25,0.50,0.50}{##1}}}
\expandafter\def\csname PY@tok@kc\endcsname{\let\PY@bf=\textbf\def\PY@tc##1{\textcolor[rgb]{0.00,0.50,0.00}{##1}}}
\expandafter\def\csname PY@tok@c\endcsname{\let\PY@it=\textit\def\PY@tc##1{\textcolor[rgb]{0.25,0.50,0.50}{##1}}}
\expandafter\def\csname PY@tok@mf\endcsname{\def\PY@tc##1{\textcolor[rgb]{0.40,0.40,0.40}{##1}}}
\expandafter\def\csname PY@tok@err\endcsname{\def\PY@bc##1{\setlength{\fboxsep}{0pt}\fcolorbox[rgb]{1.00,0.00,0.00}{1,1,1}{\strut ##1}}}
\expandafter\def\csname PY@tok@kd\endcsname{\let\PY@bf=\textbf\def\PY@tc##1{\textcolor[rgb]{0.00,0.50,0.00}{##1}}}
\expandafter\def\csname PY@tok@ss\endcsname{\def\PY@tc##1{\textcolor[rgb]{0.10,0.09,0.49}{##1}}}
\expandafter\def\csname PY@tok@sr\endcsname{\def\PY@tc##1{\textcolor[rgb]{0.73,0.40,0.53}{##1}}}
\expandafter\def\csname PY@tok@mo\endcsname{\def\PY@tc##1{\textcolor[rgb]{0.40,0.40,0.40}{##1}}}
\expandafter\def\csname PY@tok@kn\endcsname{\let\PY@bf=\textbf\def\PY@tc##1{\textcolor[rgb]{0.00,0.50,0.00}{##1}}}
\expandafter\def\csname PY@tok@mi\endcsname{\def\PY@tc##1{\textcolor[rgb]{0.40,0.40,0.40}{##1}}}
\expandafter\def\csname PY@tok@gp\endcsname{\let\PY@bf=\textbf\def\PY@tc##1{\textcolor[rgb]{0.00,0.00,0.50}{##1}}}
\expandafter\def\csname PY@tok@o\endcsname{\def\PY@tc##1{\textcolor[rgb]{0.40,0.40,0.40}{##1}}}
\expandafter\def\csname PY@tok@kr\endcsname{\let\PY@bf=\textbf\def\PY@tc##1{\textcolor[rgb]{0.00,0.50,0.00}{##1}}}
\expandafter\def\csname PY@tok@s\endcsname{\def\PY@tc##1{\textcolor[rgb]{0.73,0.13,0.13}{##1}}}
\expandafter\def\csname PY@tok@kp\endcsname{\def\PY@tc##1{\textcolor[rgb]{0.00,0.50,0.00}{##1}}}
\expandafter\def\csname PY@tok@w\endcsname{\def\PY@tc##1{\textcolor[rgb]{0.73,0.73,0.73}{##1}}}
\expandafter\def\csname PY@tok@kt\endcsname{\def\PY@tc##1{\textcolor[rgb]{0.69,0.00,0.25}{##1}}}
\expandafter\def\csname PY@tok@ow\endcsname{\let\PY@bf=\textbf\def\PY@tc##1{\textcolor[rgb]{0.67,0.13,1.00}{##1}}}
\expandafter\def\csname PY@tok@sb\endcsname{\def\PY@tc##1{\textcolor[rgb]{0.73,0.13,0.13}{##1}}}
\expandafter\def\csname PY@tok@k\endcsname{\let\PY@bf=\textbf\def\PY@tc##1{\textcolor[rgb]{0.00,0.50,0.00}{##1}}}
\expandafter\def\csname PY@tok@se\endcsname{\let\PY@bf=\textbf\def\PY@tc##1{\textcolor[rgb]{0.73,0.40,0.13}{##1}}}
\expandafter\def\csname PY@tok@sd\endcsname{\let\PY@it=\textit\def\PY@tc##1{\textcolor[rgb]{0.73,0.13,0.13}{##1}}}

\def\PYZbs{\char`\\}
\def\PYZus{\char`\_}
\def\PYZob{\char`\{}
\def\PYZcb{\char`\}}
\def\PYZca{\char`\^}
\def\PYZam{\char`\&}
\def\PYZlt{\char`\<}
\def\PYZgt{\char`\>}
\def\PYZsh{\char`\#}
\def\PYZpc{\char`\%}
\def\PYZdl{\char`\$}
\def\PYZhy{\char`\-}
\def\PYZsq{\char`\'}
\def\PYZdq{\char`\"}
\def\PYZti{\char`\~}
% for compatibility with earlier versions
\def\PYZat{@}
\def\PYZlb{[}
\def\PYZrb{]}
\makeatother


    % Exact colors from NB
    \definecolor{incolor}{rgb}{0.0, 0.0, 0.5}
    \definecolor{outcolor}{rgb}{0.545, 0.0, 0.0}



    
    % Prevent overflowing lines due to hard-to-break entities
    \sloppy 
    % Setup hyperref package
    \hypersetup{
      breaklinks=true,  % so long urls are correctly broken across lines
      colorlinks=true,
      urlcolor=blue,
      linkcolor=darkorange,
      citecolor=darkgreen,
      }
    % Slightly bigger margins than the latex defaults
    
    \geometry{verbose,tmargin=1in,bmargin=1in,lmargin=1in,rmargin=1in}
    
    

    \begin{document}
    
    
    \maketitle
    
    

    
    \section{An Introduction to Numerical Computing with Python}

While the Python language is an excellent tool for general-purpose
programming, with a highly readable syntax, rich and powerful data types
(strings, lists, sets, dictionaries, arbitrary length integers, etc) and
a very comprehensive standard library, it was not designed specifically
for mathematical and scientific computing. Neither the language nor its
standard library have facilities for the efficient representation of
multidimensional datasets, tools for linear algebra and general matrix
manipulations (an essential building block of virtually all technical
computing), nor any data visualization facilities.

In particular, Python lists are very flexible containers that can be
nested arbitrarily deep and which can hold any Python object in them,
but they are poorly suited to represent efficiently common mathematical
constructs like vectors and matrices. In contrast, much of our modern
heritage of scientific computing has been built on top of libraries
written in the Fortran language, which has native support for vectors
and matrices as well as a library of mathematical functions that can
efficiently operate on entire arrays at once.

    \subsection{Basics of Numpy arrays}

We now turn our attention to the Numpy library, which forms the base
layer for the entire `scipy ecosystem'. Once you have installed numpy,
you can import it as

    \begin{Verbatim}[commandchars=\\\{\}]
{\color{incolor}In [{\color{incolor}1}]:} \PY{k+kn}{import} \PY{n+nn}{numpy} \PY{k+kn}{as} \PY{n+nn}{np}
\end{Verbatim}

    As mentioned above, the main object provided by numpy is a powerful
array. We'll start by exploring how the numpy array differs from Python
lists. We start by creating a simple list and an array with the same
contents of the list:

    \begin{Verbatim}[commandchars=\\\{\}]
{\color{incolor}In [{\color{incolor}3}]:} \PY{n}{lst} \PY{o}{=} \PY{p}{[}\PY{l+m+mi}{10}\PY{p}{,} \PY{l+m+mi}{20}\PY{p}{,} \PY{l+m+mi}{30}\PY{p}{,} \PY{l+m+mi}{40}\PY{p}{]}
        \PY{n}{arr} \PY{o}{=} \PY{n}{np}\PY{o}{.}\PY{n}{array}\PY{p}{(}\PY{p}{[}\PY{l+m+mi}{10}\PY{p}{,} \PY{l+m+mi}{20}\PY{p}{,} \PY{l+m+mi}{30}\PY{p}{,} \PY{l+m+mi}{40}\PY{p}{]}\PY{p}{)}
        \PY{k}{print} \PY{n}{lst}
        \PY{k}{print} \PY{n}{arr}
\end{Verbatim}

    \begin{Verbatim}[commandchars=\\\{\}]
[10, 20, 30, 40]
[10 20 30 40]
    \end{Verbatim}

    \subsubsection{Element indexing}

Elements of a one-dimensional array are accessed with the same syntax as
a list:

    \begin{Verbatim}[commandchars=\\\{\}]
{\color{incolor}In [{\color{incolor}4}]:} \PY{n}{lst}\PY{p}{[}\PY{l+m+mi}{0}\PY{p}{]}
\end{Verbatim}

            \begin{Verbatim}[commandchars=\\\{\}]
{\color{outcolor}Out[{\color{outcolor}4}]:} 10
\end{Verbatim}
        
    \begin{Verbatim}[commandchars=\\\{\}]
{\color{incolor}In [{\color{incolor}1}]:} \PY{n}{arr}\PY{p}{[}\PY{l+m+mi}{0}\PY{p}{]}
\end{Verbatim}

    \begin{Verbatim}[commandchars=\\\{\}]

        ---------------------------------------------------------------------------
    NameError                                 Traceback (most recent call last)

        <ipython-input-1-1c936cf4bf16> in <module>()
    ----> 1 arr[0]
    

        NameError: name 'arr' is not defined

    \end{Verbatim}

    \begin{Verbatim}[commandchars=\\\{\}]
{\color{incolor}In [{\color{incolor}6}]:} \PY{n}{arr}\PY{p}{[}\PY{o}{\PYZhy{}}\PY{l+m+mi}{1}\PY{p}{]}
\end{Verbatim}

            \begin{Verbatim}[commandchars=\\\{\}]
{\color{outcolor}Out[{\color{outcolor}6}]:} 40
\end{Verbatim}
        
    \begin{Verbatim}[commandchars=\\\{\}]
{\color{incolor}In [{\color{incolor}7}]:} \PY{n}{arr}\PY{p}{[}\PY{l+m+mi}{2}\PY{p}{:}\PY{p}{]}
\end{Verbatim}

            \begin{Verbatim}[commandchars=\\\{\}]
{\color{outcolor}Out[{\color{outcolor}7}]:} array([30, 40])
\end{Verbatim}
        
    \subsubsection{Differences between arrays and lists}

The first difference to note between lists and arrays is that arrays are
\emph{homogeneous}; i.e.~all elements of an array must be of the same
type. In contrast, lists can contain elements of arbitrary type. For
example, we can change the last element in our list above to be a
string:

    \begin{Verbatim}[commandchars=\\\{\}]
{\color{incolor}In [{\color{incolor}8}]:} \PY{n}{lst}\PY{p}{[}\PY{o}{\PYZhy{}}\PY{l+m+mi}{1}\PY{p}{]} \PY{o}{=} \PY{l+s}{\PYZsq{}}\PY{l+s}{a string inside a list}\PY{l+s}{\PYZsq{}}
        \PY{n}{lst}
\end{Verbatim}

            \begin{Verbatim}[commandchars=\\\{\}]
{\color{outcolor}Out[{\color{outcolor}8}]:} [10, 20, 30, 'a string inside a list']
\end{Verbatim}
        
    but the same can not be done with an array, as we get an error message:

    \begin{Verbatim}[commandchars=\\\{\}]
{\color{incolor}In [{\color{incolor}9}]:} \PY{n}{arr}\PY{p}{[}\PY{o}{\PYZhy{}}\PY{l+m+mi}{1}\PY{p}{]} \PY{o}{=} \PY{l+s}{\PYZsq{}}\PY{l+s}{a string inside an array}\PY{l+s}{\PYZsq{}}
\end{Verbatim}

    \begin{Verbatim}[commandchars=\\\{\}]

        ---------------------------------------------------------------------------
    ValueError                                Traceback (most recent call last)

        <ipython-input-9-29c0bfa5fa8a> in <module>()
    ----> 1 arr[-1] = 'a string inside an array'
    

        ValueError: invalid literal for long() with base 10: 'a string inside an array'

    \end{Verbatim}

    Caveat, it can be done, but really \emph{don't do it}, lists are
generally better at non-homogeonous collections:

    \begin{Verbatim}[commandchars=\\\{\}]
{\color{incolor}In [{\color{incolor}1}]:} \PY{n}{np}\PY{o}{.}\PY{n}{array}\PY{p}{(}\PY{p}{[}\PY{l+m+mi}{10}\PY{p}{,} \PY{l+m+mi}{20}\PY{p}{,} \PY{l+m+mi}{30}\PY{p}{,} \PY{l+s}{\PYZsq{}}\PY{l+s}{a string inside a list}\PY{l+s}{\PYZsq{}}\PY{p}{]}\PY{p}{)}
\end{Verbatim}

            \begin{Verbatim}[commandchars=\\\{\}]
{\color{outcolor}Out[{\color{outcolor}1}]:} array(['10', '20', '30', 'a string inside a list'], 
              dtype='|S22')
\end{Verbatim}
        
    \begin{Verbatim}[commandchars=\\\{\}]
{\color{incolor}In [{\color{incolor}2}]:} \PY{n}{np}\PY{o}{.}\PY{n}{array}\PY{p}{(}\PY{p}{[}\PY{l+m+mi}{10}\PY{p}{,} \PY{l+m+mi}{20}\PY{p}{,} \PY{l+m+mi}{30}\PY{p}{,} \PY{l+s}{\PYZsq{}}\PY{l+s}{a string inside a list}\PY{l+s}{\PYZsq{}}\PY{p}{]}\PY{p}{,} \PY{n}{dtype}\PY{o}{=}\PY{n+nb}{object}\PY{p}{)}
\end{Verbatim}

            \begin{Verbatim}[commandchars=\\\{\}]
{\color{outcolor}Out[{\color{outcolor}2}]:} array([10, 20, 30, 'a string inside a list'], dtype=object)
\end{Verbatim}
        
    \subsubsection{Array Attributes}

The information about the type of an array is contained in its
\emph{dtype} attribute:

    \begin{Verbatim}[commandchars=\\\{\}]
{\color{incolor}In [{\color{incolor}10}]:} \PY{n}{arr}\PY{o}{.}\PY{n}{dtype}
\end{Verbatim}

            \begin{Verbatim}[commandchars=\\\{\}]
{\color{outcolor}Out[{\color{outcolor}10}]:} dtype('int64')
\end{Verbatim}
        
    Once an array has been created, its dtype is fixed and it can only store
elements of the same type. For this example where the dtype is integer,
if we store a floating point number it will be automatically converted
into an integer:

    \begin{Verbatim}[commandchars=\\\{\}]
{\color{incolor}In [{\color{incolor}11}]:} \PY{n}{arr}\PY{p}{[}\PY{o}{\PYZhy{}}\PY{l+m+mi}{1}\PY{p}{]} \PY{o}{=} \PY{l+m+mf}{1.234}
         \PY{n}{arr}
\end{Verbatim}

            \begin{Verbatim}[commandchars=\\\{\}]
{\color{outcolor}Out[{\color{outcolor}11}]:} array([10, 20, 30,  1])
\end{Verbatim}
        
    \subsubsection{Creating Arrays}

Above we created an array from an existing list; now let us now see
other ways in which we can create arrays, which we'll illustrate next. A
common need is to have an array initialized with a constant value, and
very often this value is 0 or 1 (suitable as starting value for additive
and multiplicative loops respectively); \texttt{zeros} creates arrays of
all zeros, with any desired dtype:

    \begin{Verbatim}[commandchars=\\\{\}]
{\color{incolor}In [{\color{incolor}12}]:} \PY{n}{np}\PY{o}{.}\PY{n}{zeros}\PY{p}{(}\PY{l+m+mi}{5}\PY{p}{,} \PY{n}{dtype}\PY{o}{=}\PY{n+nb}{float}\PY{p}{)}
\end{Verbatim}

            \begin{Verbatim}[commandchars=\\\{\}]
{\color{outcolor}Out[{\color{outcolor}12}]:} array([ 0.,  0.,  0.,  0.,  0.])
\end{Verbatim}
        
    \begin{Verbatim}[commandchars=\\\{\}]
{\color{incolor}In [{\color{incolor}13}]:} \PY{n}{np}\PY{o}{.}\PY{n}{zeros}\PY{p}{(}\PY{l+m+mi}{3}\PY{p}{,} \PY{n}{dtype}\PY{o}{=}\PY{n+nb}{int}\PY{p}{)}
\end{Verbatim}

            \begin{Verbatim}[commandchars=\\\{\}]
{\color{outcolor}Out[{\color{outcolor}13}]:} array([0, 0, 0])
\end{Verbatim}
        
    \begin{Verbatim}[commandchars=\\\{\}]
{\color{incolor}In [{\color{incolor}14}]:} \PY{n}{np}\PY{o}{.}\PY{n}{zeros}\PY{p}{(}\PY{l+m+mi}{3}\PY{p}{,} \PY{n}{dtype}\PY{o}{=}\PY{n+nb}{complex}\PY{p}{)}
\end{Verbatim}

            \begin{Verbatim}[commandchars=\\\{\}]
{\color{outcolor}Out[{\color{outcolor}14}]:} array([ 0.+0.j,  0.+0.j,  0.+0.j])
\end{Verbatim}
        
    and similarly for \texttt{ones}:

    \begin{Verbatim}[commandchars=\\\{\}]
{\color{incolor}In [{\color{incolor}15}]:} \PY{k}{print} \PY{l+s}{\PYZsq{}}\PY{l+s}{5 ones:}\PY{l+s}{\PYZsq{}}\PY{p}{,} \PY{n}{np}\PY{o}{.}\PY{n}{ones}\PY{p}{(}\PY{l+m+mi}{5}\PY{p}{)}
\end{Verbatim}

    \begin{Verbatim}[commandchars=\\\{\}]
5 ones: [ 1.  1.  1.  1.  1.]
    \end{Verbatim}

    If we want an array initialized with an arbitrary value, we can create
an empty array and then use the fill method to put the value we want
into the array:

    \begin{Verbatim}[commandchars=\\\{\}]
{\color{incolor}In [{\color{incolor}16}]:} \PY{n}{a} \PY{o}{=} \PY{n}{np}\PY{o}{.}\PY{n}{empty}\PY{p}{(}\PY{l+m+mi}{4}\PY{p}{)}
         \PY{n}{a}\PY{o}{.}\PY{n}{fill}\PY{p}{(}\PY{l+m+mf}{5.5}\PY{p}{)}
         \PY{n}{a}
\end{Verbatim}

            \begin{Verbatim}[commandchars=\\\{\}]
{\color{outcolor}Out[{\color{outcolor}16}]:} array([ 5.5,  5.5,  5.5,  5.5])
\end{Verbatim}
        
    Alternatives, such as:

\begin{itemize}
\itemsep1pt\parskip0pt\parsep0pt
\item
  \texttt{np.ones(4) * 5.5}
\item
  \texttt{np.zeros(4) + 5.5}
\end{itemize}

are generally less efficient, but are also reasonable.

    \subsubsection{Filling arrays with sequences}

Numpy also offers the \texttt{arange} function, which works like the
builtin \texttt{range} but returns an array instead of a list:

    \begin{Verbatim}[commandchars=\\\{\}]
{\color{incolor}In [{\color{incolor}20}]:} \PY{n}{np}\PY{o}{.}\PY{n}{arange}\PY{p}{(}\PY{l+m+mi}{5}\PY{p}{,} \PY{l+m+mi}{10}\PY{p}{,} \PY{l+m+mf}{0.1}\PY{p}{)}
\end{Verbatim}

            \begin{Verbatim}[commandchars=\\\{\}]
{\color{outcolor}Out[{\color{outcolor}20}]:} array([ 5. ,  5.1,  5.2,  5.3,  5.4,  5.5,  5.6,  5.7,  5.8,  5.9,  6. ,
                 6.1,  6.2,  6.3,  6.4,  6.5,  6.6,  6.7,  6.8,  6.9,  7. ,  7.1,
                 7.2,  7.3,  7.4,  7.5,  7.6,  7.7,  7.8,  7.9,  8. ,  8.1,  8.2,
                 8.3,  8.4,  8.5,  8.6,  8.7,  8.8,  8.9,  9. ,  9.1,  9.2,  9.3,
                 9.4,  9.5,  9.6,  9.7,  9.8,  9.9])
\end{Verbatim}
        
    and the \texttt{linspace} and \texttt{logspace} functions to create
linearly and logarithmically-spaced grids respectively, with a fixed
number of points and including both ends of the specified interval:

    \begin{Verbatim}[commandchars=\\\{\}]
{\color{incolor}In [{\color{incolor}3}]:} \PY{k}{print} \PY{l+s}{\PYZdq{}}\PY{l+s}{A linear grid between 0 and 1:}\PY{l+s}{\PYZdq{}}
        \PY{k}{print} \PY{n}{np}\PY{o}{.}\PY{n}{linspace}\PY{p}{(}\PY{l+m+mi}{0}\PY{p}{,} \PY{l+m+mi}{1}\PY{p}{,} \PY{l+m+mi}{5}\PY{p}{)}
\end{Verbatim}

    \begin{Verbatim}[commandchars=\\\{\}]
A linear grid between 0 and 1:
[ 0.    0.25  0.5   0.75  1.  ]
    \end{Verbatim}

    \begin{Verbatim}[commandchars=\\\{\}]
{\color{incolor}In [{\color{incolor}9}]:} \PY{k}{print} \PY{l+s}{\PYZdq{}}\PY{l+s}{A logarithmic grid between 10**2 and 10**4:}\PY{l+s}{\PYZdq{}}
        \PY{k}{print} \PY{n}{np}\PY{o}{.}\PY{n}{logspace}\PY{p}{(}\PY{l+m+mi}{2}\PY{p}{,} \PY{l+m+mi}{4}\PY{p}{,} \PY{l+m+mi}{3}\PY{p}{)}
\end{Verbatim}

    \begin{Verbatim}[commandchars=\\\{\}]
A logarithmic grid between 10**1 and 10**3:
[   100.   1000.  10000.]
    \end{Verbatim}

    \subsubsection{Creating random arrays}

Finally, it is often useful to create arrays with random numbers that
follow a specific distribution. The \texttt{np.random} module contains a
number of functions that can be used to this effect, for example this
will produce an array of 5 random samples taken from a standard normal
distribution (0 mean and variance 1):

    \begin{Verbatim}[commandchars=\\\{\}]
{\color{incolor}In [{\color{incolor}23}]:} \PY{n}{np}\PY{o}{.}\PY{n}{random}\PY{o}{.}\PY{n}{randn}\PY{p}{(}\PY{l+m+mi}{5}\PY{p}{)}
\end{Verbatim}

            \begin{Verbatim}[commandchars=\\\{\}]
{\color{outcolor}Out[{\color{outcolor}23}]:} array([-0.84334417, -0.40047238,  0.26216058,  1.42346312, -1.05209843])
\end{Verbatim}
        
    whereas this will also give 5 samples, but from a normal distribution
with a mean of 10 and a variance of 3:

    \begin{Verbatim}[commandchars=\\\{\}]
{\color{incolor}In [{\color{incolor}25}]:} \PY{n}{norm10} \PY{o}{=} \PY{n}{np}\PY{o}{.}\PY{n}{random}\PY{o}{.}\PY{n}{normal}\PY{p}{(}\PY{l+m+mi}{10}\PY{p}{,} \PY{l+m+mi}{3}\PY{p}{,} \PY{l+m+mi}{5}\PY{p}{)}
         \PY{n}{norm10}
\end{Verbatim}

            \begin{Verbatim}[commandchars=\\\{\}]
{\color{outcolor}Out[{\color{outcolor}25}]:} array([  5.69464342,  11.3800105 ,   6.90729538,  10.4329192 ,  16.04373525])
\end{Verbatim}
        
    \subsection{Indexing with other arrays}

Above we saw how to index arrays with single numbers and slices, just
like Python lists. But arrays allow for a more sophisticated kind of
indexing which is very powerful: you can index an array with another
array, and in particular with an array of boolean values. This is
particluarly useful to extract information from an array that matches a
certain condition.

Consider for example that in the array \texttt{norm10} we want to
replace all values above 9 with the value 0. We can do so by first
finding the \emph{mask} that indicates where this condition is true or
false:

    \begin{Verbatim}[commandchars=\\\{\}]
{\color{incolor}In [{\color{incolor}26}]:} \PY{n}{mask} \PY{o}{=} \PY{n}{norm10} \PY{o}{\PYZgt{}} \PY{l+m+mi}{9}
         \PY{n}{mask}
\end{Verbatim}

            \begin{Verbatim}[commandchars=\\\{\}]
{\color{outcolor}Out[{\color{outcolor}26}]:} array([False,  True, False,  True,  True], dtype=bool)
\end{Verbatim}
        
    Now that we have this mask, we can use it to either read those values or
to reset them to 0:

    \begin{Verbatim}[commandchars=\\\{\}]
{\color{incolor}In [{\color{incolor}28}]:} \PY{k}{print} \PY{l+s}{\PYZsq{}}\PY{l+s}{Values above 9:}\PY{l+s}{\PYZsq{}}\PY{p}{,} \PY{n}{norm10}\PY{p}{[}\PY{n}{norm10} \PY{o}{\PYZgt{}} \PY{l+m+mi}{9}\PY{p}{]}
\end{Verbatim}

    \begin{Verbatim}[commandchars=\\\{\}]
Values above 9: [ 11.3800105   10.4329192   16.04373525]
    \end{Verbatim}

    \begin{Verbatim}[commandchars=\\\{\}]
{\color{incolor}In [{\color{incolor}29}]:} \PY{k}{print} \PY{l+s}{\PYZsq{}}\PY{l+s}{Resetting all values above 9 to 0...}\PY{l+s}{\PYZsq{}}
         \PY{n}{norm10}\PY{p}{[}\PY{n}{mask}\PY{p}{]} \PY{o}{=} \PY{l+m+mi}{0}
         \PY{k}{print} \PY{n}{norm10}
\end{Verbatim}

    \begin{Verbatim}[commandchars=\\\{\}]
Resetting all values above 9 to 0\ldots
[ 5.69464342  0.          6.90729538  0.          0.        ]
    \end{Verbatim}

    \subsection{Arrays with more than one dimension}

Up until now all our examples have used one-dimensional arrays. But
Numpy can create arrays of aribtrary dimensions, and all the methods
illustrated in the previous section work with more than one dimension.
For example, a list of lists can be used to initialize a two dimensional
array:

    \begin{Verbatim}[commandchars=\\\{\}]
{\color{incolor}In [{\color{incolor}31}]:} \PY{n}{lst2} \PY{o}{=} \PY{p}{[}\PY{p}{[}\PY{l+m+mi}{1}\PY{p}{,} \PY{l+m+mi}{2}\PY{p}{]}\PY{p}{,} \PY{p}{[}\PY{l+m+mi}{3}\PY{p}{,} \PY{l+m+mi}{4}\PY{p}{]}\PY{p}{]}
         \PY{n}{arr2} \PY{o}{=} \PY{n}{np}\PY{o}{.}\PY{n}{array}\PY{p}{(}\PY{p}{[}\PY{p}{[}\PY{l+m+mi}{1}\PY{p}{,} \PY{l+m+mi}{2}\PY{p}{]}\PY{p}{,} \PY{p}{[}\PY{l+m+mi}{3}\PY{p}{,} \PY{l+m+mi}{4}\PY{p}{]}\PY{p}{]}\PY{p}{)}
         \PY{n}{arr2}
         \PY{n}{arr2}\PY{o}{.}\PY{n}{shape}
\end{Verbatim}

            \begin{Verbatim}[commandchars=\\\{\}]
{\color{outcolor}Out[{\color{outcolor}31}]:} (2, 2)
\end{Verbatim}
        
    With two-dimensional arrays we start seeing the power of numpy: while a
nested list can be indexed using repeatedly the \texttt{{[} {]}}
operator, multidimensional arrays support a much more natural indexing
syntax with a single \texttt{{[} {]}} and a set of indices separated by
commas:

    \begin{Verbatim}[commandchars=\\\{\}]
{\color{incolor}In [{\color{incolor}33}]:} \PY{k}{print} \PY{n}{lst2}\PY{p}{[}\PY{l+m+mi}{0}\PY{p}{]}\PY{p}{[}\PY{l+m+mi}{1}\PY{p}{]}
         \PY{k}{print} \PY{n}{arr2}\PY{p}{[}\PY{l+m+mi}{0}\PY{p}{,} \PY{l+m+mi}{1}\PY{p}{]}
\end{Verbatim}

    \begin{Verbatim}[commandchars=\\\{\}]
2
2
    \end{Verbatim}

    \subsection{Arrays with more than one dimension}

Most of the array creation functions listed above can be used with more
than one dimension, for example:

    \begin{Verbatim}[commandchars=\\\{\}]
{\color{incolor}In [{\color{incolor}34}]:} \PY{n}{np}\PY{o}{.}\PY{n}{zeros}\PY{p}{(}\PY{p}{(}\PY{l+m+mi}{2}\PY{p}{,} \PY{l+m+mi}{3}\PY{p}{)}\PY{p}{)}
\end{Verbatim}

            \begin{Verbatim}[commandchars=\\\{\}]
{\color{outcolor}Out[{\color{outcolor}34}]:} array([[ 0.,  0.,  0.],
                [ 0.,  0.,  0.]])
\end{Verbatim}
        
    \begin{Verbatim}[commandchars=\\\{\}]
{\color{incolor}In [{\color{incolor}35}]:} \PY{n}{np}\PY{o}{.}\PY{n}{random}\PY{o}{.}\PY{n}{normal}\PY{p}{(}\PY{l+m+mi}{10}\PY{p}{,} \PY{l+m+mi}{3}\PY{p}{,} \PY{p}{(}\PY{l+m+mi}{2}\PY{p}{,} \PY{l+m+mi}{4}\PY{p}{)}\PY{p}{)}
\end{Verbatim}

            \begin{Verbatim}[commandchars=\\\{\}]
{\color{outcolor}Out[{\color{outcolor}35}]:} array([[ 11.21935429,   6.95937781,  12.22843875,   8.29844811],
                [ 12.47104408,   9.60844803,   9.21547789,  15.70517319]])
\end{Verbatim}
        
    In fact, the shape of an array can be changed at any time, as long as
the total number of elements is unchanged. For example, if we want a 2x4
array with numbers increasing from 0, the easiest way to create it is:

    \begin{Verbatim}[commandchars=\\\{\}]
{\color{incolor}In [{\color{incolor}10}]:} \PY{n}{arr} \PY{o}{=} \PY{n}{np}\PY{o}{.}\PY{n}{arange}\PY{p}{(}\PY{l+m+mi}{8}\PY{p}{)}\PY{o}{.}\PY{n}{reshape}\PY{p}{(}\PY{l+m+mi}{2}\PY{p}{,} \PY{l+m+mi}{4}\PY{p}{)}
         \PY{k}{print} \PY{n}{arr}
\end{Verbatim}

    \begin{Verbatim}[commandchars=\\\{\}]
[[0 1 2 3]
 [4 5 6 7]]
    \end{Verbatim}

    \subsubsection{Views, not Copies}

Note that reshaping (like most numpy operations), wherever possible,
provides a \textbf{view} of \emph{the same memory}:

    \begin{Verbatim}[commandchars=\\\{\}]
{\color{incolor}In [{\color{incolor}30}]:} \PY{n}{arr} \PY{o}{=} \PY{n}{np}\PY{o}{.}\PY{n}{arange}\PY{p}{(}\PY{l+m+mi}{8}\PY{p}{)}
         \PY{n}{arr\PYZus{}view} \PY{o}{=} \PY{n}{arr}\PY{o}{.}\PY{n}{reshape}\PY{p}{(}\PY{l+m+mi}{2}\PY{p}{,} \PY{l+m+mi}{4}\PY{p}{)}
         
         \PY{k}{print} \PY{l+s}{\PYZsq{}}\PY{l+s}{Before}\PY{l+s+se}{\PYZbs{}n}\PY{l+s}{\PYZsq{}}\PY{p}{,} \PY{n}{arr\PYZus{}view}
         
         \PY{n}{arr}\PY{p}{[}\PY{l+m+mi}{0}\PY{p}{]} \PY{o}{=} \PY{l+m+mi}{1000}
         
         \PY{k}{print} \PY{l+s}{\PYZsq{}}\PY{l+s}{After}\PY{l+s+se}{\PYZbs{}n}\PY{l+s}{\PYZsq{}}\PY{p}{,} \PY{n}{arr\PYZus{}view}
\end{Verbatim}

    \begin{Verbatim}[commandchars=\\\{\}]
Before
[[0 1 2 3]
 [4 5 6 7]]
After
[[1000    1    2    3]
 [   4    5    6    7]]
    \end{Verbatim}

    This lack of copying allows for very efficient vectorized operations,
but this power should be used responsibly.

If in doubt, you can always copy the data to a different block of memory
with the copy method.

    \subsubsection{Slices}

With multidimensional arrays, you can also use slices, and you can mix
and match slices and single indices in the different dimensions:

    \begin{Verbatim}[commandchars=\\\{\}]
{\color{incolor}In [{\color{incolor}31}]:} \PY{n}{arr} \PY{o}{=} \PY{n}{np}\PY{o}{.}\PY{n}{arange}\PY{p}{(}\PY{l+m+mi}{8}\PY{p}{)}\PY{o}{.}\PY{n}{reshape}\PY{p}{(}\PY{l+m+mi}{2}\PY{p}{,} \PY{l+m+mi}{4}\PY{p}{)}
         \PY{k}{print} \PY{n}{arr}
         \PY{k}{print} \PY{l+s}{\PYZsq{}}\PY{l+s}{Slicing in the second row:}\PY{l+s}{\PYZsq{}}\PY{p}{,} \PY{n}{arr}\PY{p}{[}\PY{l+m+mi}{1}\PY{p}{,} \PY{l+m+mi}{2}\PY{p}{:}\PY{l+m+mi}{4}\PY{p}{]}
         \PY{k}{print} \PY{l+s}{\PYZsq{}}\PY{l+s}{All rows, third column   :}\PY{l+s}{\PYZsq{}}\PY{p}{,} \PY{n}{arr}\PY{p}{[}\PY{p}{:}\PY{p}{,} \PY{l+m+mi}{2}\PY{p}{]}
\end{Verbatim}

    \begin{Verbatim}[commandchars=\\\{\}]
[[0 1 2 3]
 [4 5 6 7]]
Slicing in the second row: [6 7]
All rows, third column   : [2 6]
    \end{Verbatim}

    If you only provide one index, then you will get an array with one less
dimension containing that row:

    \begin{Verbatim}[commandchars=\\\{\}]
{\color{incolor}In [{\color{incolor}45}]:} \PY{k}{print} \PY{l+s}{\PYZsq{}}\PY{l+s}{First row:  }\PY{l+s}{\PYZsq{}}\PY{p}{,} \PY{n}{arr}\PY{p}{[}\PY{l+m+mi}{0}\PY{p}{]}
         \PY{k}{print} \PY{l+s}{\PYZsq{}}\PY{l+s}{Second row: }\PY{l+s}{\PYZsq{}}\PY{p}{,} \PY{n}{arr}\PY{p}{[}\PY{l+m+mi}{1}\PY{p}{]}
\end{Verbatim}

    \begin{Verbatim}[commandchars=\\\{\}]
First row:   [0 1 2 3]
Second row:  [4 5 6 7]
    \end{Verbatim}

    \subsection{Array Properties and Methods}

Now that we have seen how to create arrays with more than one dimension,
it's a good idea to look at some of the most useful properties and
methods that arrays have. The following provide basic information about
the size, shape and data in the array:

    \begin{Verbatim}[commandchars=\\\{\}]
{\color{incolor}In [{\color{incolor}48}]:} \PY{k}{print} \PY{l+s}{\PYZsq{}}\PY{l+s}{Data type                :}\PY{l+s}{\PYZsq{}}\PY{p}{,} \PY{n}{arr}\PY{o}{.}\PY{n}{dtype}
         \PY{k}{print} \PY{l+s}{\PYZsq{}}\PY{l+s}{Total number of elements :}\PY{l+s}{\PYZsq{}}\PY{p}{,} \PY{n}{arr}\PY{o}{.}\PY{n}{size}
         \PY{k}{print} \PY{l+s}{\PYZsq{}}\PY{l+s}{Number of dimensions     :}\PY{l+s}{\PYZsq{}}\PY{p}{,} \PY{n}{arr}\PY{o}{.}\PY{n}{ndim}
         \PY{k}{print} \PY{l+s}{\PYZsq{}}\PY{l+s}{Shape (dimensionality)   :}\PY{l+s}{\PYZsq{}}\PY{p}{,} \PY{n}{arr}\PY{o}{.}\PY{n}{shape}
         \PY{k}{print} \PY{l+s}{\PYZsq{}}\PY{l+s}{Memory used (in bytes)   :}\PY{l+s}{\PYZsq{}}\PY{p}{,} \PY{n}{arr}\PY{o}{.}\PY{n}{nbytes}
\end{Verbatim}

    \begin{Verbatim}[commandchars=\\\{\}]
Data type                : int64
Total number of elements : 8
Number of dimensions     : 2
Shape (dimensionality)   : (2, 4)
Memory used (in bytes)   : 64
    \end{Verbatim}

    Arrays also have many useful methods, some especially useful ones are:

    \begin{Verbatim}[commandchars=\\\{\}]
{\color{incolor}In [{\color{incolor}49}]:} \PY{k}{print} \PY{l+s}{\PYZsq{}}\PY{l+s}{Minimum and maximum             :}\PY{l+s}{\PYZsq{}}\PY{p}{,} \PY{n}{arr}\PY{o}{.}\PY{n}{min}\PY{p}{(}\PY{p}{)}\PY{p}{,} \PY{n}{arr}\PY{o}{.}\PY{n}{max}\PY{p}{(}\PY{p}{)}
         \PY{k}{print} \PY{l+s}{\PYZsq{}}\PY{l+s}{Sum and product of all elements :}\PY{l+s}{\PYZsq{}}\PY{p}{,} \PY{n}{arr}\PY{o}{.}\PY{n}{sum}\PY{p}{(}\PY{p}{)}\PY{p}{,} \PY{n}{arr}\PY{o}{.}\PY{n}{prod}\PY{p}{(}\PY{p}{)}
         \PY{k}{print} \PY{l+s}{\PYZsq{}}\PY{l+s}{Mean and standard deviation     :}\PY{l+s}{\PYZsq{}}\PY{p}{,} \PY{n}{arr}\PY{o}{.}\PY{n}{mean}\PY{p}{(}\PY{p}{)}\PY{p}{,} \PY{n}{arr}\PY{o}{.}\PY{n}{std}\PY{p}{(}\PY{p}{)}
\end{Verbatim}

    \begin{Verbatim}[commandchars=\\\{\}]
Minimum and maximum             : 0 7
Sum and product of all elements : 28 0
Mean and standard deviation     : 3.5 2.29128784748
    \end{Verbatim}

    For these methods, the above operations area all computed on all the
elements of the array. But for a multidimensional array, it's possible
to do the computation along a single dimension, by passing the
\texttt{axis} parameter; for example:

    \begin{Verbatim}[commandchars=\\\{\}]
{\color{incolor}In [{\color{incolor}50}]:} \PY{k}{print} \PY{l+s}{\PYZsq{}}\PY{l+s}{For the following array:}\PY{l+s+se}{\PYZbs{}n}\PY{l+s}{\PYZsq{}}\PY{p}{,} \PY{n}{arr}
         \PY{k}{print} \PY{l+s}{\PYZsq{}}\PY{l+s}{The sum of elements along the rows is    :}\PY{l+s}{\PYZsq{}}\PY{p}{,} \PY{n}{arr}\PY{o}{.}\PY{n}{sum}\PY{p}{(}\PY{n}{axis}\PY{o}{=}\PY{l+m+mi}{1}\PY{p}{)}
         \PY{k}{print} \PY{l+s}{\PYZsq{}}\PY{l+s}{The sum of elements along the columns is :}\PY{l+s}{\PYZsq{}}\PY{p}{,} \PY{n}{arr}\PY{o}{.}\PY{n}{sum}\PY{p}{(}\PY{n}{axis}\PY{o}{=}\PY{l+m+mi}{0}\PY{p}{)}
\end{Verbatim}

    \begin{Verbatim}[commandchars=\\\{\}]
For the following array:
[[0 1 2 3]
 [4 5 6 7]]
The sum of elements along the rows is    : [ 6 22]
The sum of elements along the columns is : [ 4  6  8 10]
    \end{Verbatim}

    As you can see in this example, the value of the \texttt{axis} parameter
is the dimension which will be \emph{consumed} once the operation has
been carried out. This is why to sum along the rows we use
\texttt{axis=0}.

This can be easily illustrated with an example that has more dimensions;
we create an array with 4 dimensions and shape \texttt{(3,4,5,6)} and
sum along the axis number 2 (i.e.~the \emph{third} axis, since in Python
all counts are 0-based). That consumes the dimension whose length was 5,
leaving us with a new array that has shape \texttt{(3,4,6)}:

    \begin{Verbatim}[commandchars=\\\{\}]
{\color{incolor}In [{\color{incolor}51}]:} \PY{n}{np}\PY{o}{.}\PY{n}{zeros}\PY{p}{(}\PY{p}{(}\PY{l+m+mi}{3}\PY{p}{,}\PY{l+m+mi}{4}\PY{p}{,}\PY{l+m+mi}{5}\PY{p}{,}\PY{l+m+mi}{6}\PY{p}{)}\PY{p}{)}\PY{o}{.}\PY{n}{sum}\PY{p}{(}\PY{l+m+mi}{2}\PY{p}{)}\PY{o}{.}\PY{n}{shape}
\end{Verbatim}

            \begin{Verbatim}[commandchars=\\\{\}]
{\color{outcolor}Out[{\color{outcolor}51}]:} (3, 4, 6)
\end{Verbatim}
        
    Another widely used property of arrays is the \texttt{.T} attribute,
which allows you to access the transpose of the array:

    \begin{Verbatim}[commandchars=\\\{\}]
{\color{incolor}In [{\color{incolor}52}]:} \PY{k}{print} \PY{l+s}{\PYZsq{}}\PY{l+s}{Array:}\PY{l+s+se}{\PYZbs{}n}\PY{l+s}{\PYZsq{}}\PY{p}{,} \PY{n}{arr}
         \PY{k}{print} \PY{l+s}{\PYZsq{}}\PY{l+s}{Transpose:}\PY{l+s+se}{\PYZbs{}n}\PY{l+s}{\PYZsq{}}\PY{p}{,} \PY{n}{arr}\PY{o}{.}\PY{n}{T}
\end{Verbatim}

    \begin{Verbatim}[commandchars=\\\{\}]
Array:
[[0 1 2 3]
 [4 5 6 7]]
Transpose:
[[0 4]
 [1 5]
 [2 6]
 [3 7]]
    \end{Verbatim}

    \subsection{Operating with arrays}

Arrays support all regular arithmetic operators, and the numpy library
also contains a complete collection of basic mathematical functions that
operate on arrays. It is important to remember that in general, all
operations with arrays are applied \emph{element-wise}, i.e., are
applied to all the elements of the array at the same time. Consider for
example:

    \begin{Verbatim}[commandchars=\\\{\}]
{\color{incolor}In [{\color{incolor}53}]:} \PY{n}{arr1} \PY{o}{=} \PY{n}{np}\PY{o}{.}\PY{n}{arange}\PY{p}{(}\PY{l+m+mi}{4}\PY{p}{)}
         \PY{n}{arr2} \PY{o}{=} \PY{n}{np}\PY{o}{.}\PY{n}{arange}\PY{p}{(}\PY{l+m+mi}{10}\PY{p}{,} \PY{l+m+mi}{14}\PY{p}{)}
         \PY{k}{print} \PY{n}{arr1}\PY{p}{,} \PY{l+s}{\PYZsq{}}\PY{l+s}{+}\PY{l+s}{\PYZsq{}}\PY{p}{,} \PY{n}{arr2}\PY{p}{,} \PY{l+s}{\PYZsq{}}\PY{l+s}{=}\PY{l+s}{\PYZsq{}}\PY{p}{,} \PY{n}{arr1} \PY{o}{+} \PY{n}{arr2}
\end{Verbatim}

    \begin{Verbatim}[commandchars=\\\{\}]
[0 1 2 3] + [10 11 12 13] = [10 12 14 16]
    \end{Verbatim}

    Importantly, you must remember that even the multiplication operator is
by default applied element-wise, it is \emph{not} the matrix
multiplication from linear algebra:

    \begin{Verbatim}[commandchars=\\\{\}]
{\color{incolor}In [{\color{incolor}54}]:} \PY{k}{print} \PY{n}{arr1}\PY{p}{,} \PY{l+s}{\PYZsq{}}\PY{l+s}{*}\PY{l+s}{\PYZsq{}}\PY{p}{,} \PY{n}{arr2}\PY{p}{,} \PY{l+s}{\PYZsq{}}\PY{l+s}{=}\PY{l+s}{\PYZsq{}}\PY{p}{,} \PY{n}{arr1} \PY{o}{*} \PY{n}{arr2}
\end{Verbatim}

    \begin{Verbatim}[commandchars=\\\{\}]
[0 1 2 3] * [10 11 12 13] = [ 0 11 24 39]
    \end{Verbatim}

    We may also multiply an array by a scalar:

    \begin{Verbatim}[commandchars=\\\{\}]
{\color{incolor}In [{\color{incolor}55}]:} \PY{l+m+mf}{1.5} \PY{o}{*} \PY{n}{arr1}
\end{Verbatim}

            \begin{Verbatim}[commandchars=\\\{\}]
{\color{outcolor}Out[{\color{outcolor}55}]:} array([ 0. ,  1.5,  3. ,  4.5])
\end{Verbatim}
        
    This is a first example of \textbf{broadcasting}

    \subsubsection{Broadcasting}

While the fact that numpy operates on an element-wise basis means that
in principle arrays must always match in their dimensionality, numpy
will \emph{broadcast} dimensions when possible. Here is an example of
broadcasting a scalar to a 1D array:

    \begin{Verbatim}[commandchars=\\\{\}]
{\color{incolor}In [{\color{incolor}56}]:} \PY{k}{print} \PY{n}{np}\PY{o}{.}\PY{n}{arange}\PY{p}{(}\PY{l+m+mi}{3}\PY{p}{)}
         \PY{k}{print} \PY{n}{np}\PY{o}{.}\PY{n}{arange}\PY{p}{(}\PY{l+m+mi}{3}\PY{p}{)} \PY{o}{+} \PY{l+m+mi}{5}
\end{Verbatim}

    \begin{Verbatim}[commandchars=\\\{\}]
[0 1 2]
[5 6 7]
    \end{Verbatim}

    We can also broadcast a 1D array to a 2D array, in this case adding a
vector to all rows of a matrix:

    \begin{Verbatim}[commandchars=\\\{\}]
{\color{incolor}In [{\color{incolor}57}]:} \PY{n}{np}\PY{o}{.}\PY{n}{ones}\PY{p}{(}\PY{p}{(}\PY{l+m+mi}{3}\PY{p}{,} \PY{l+m+mi}{3}\PY{p}{)}\PY{p}{)} \PY{o}{+} \PY{n}{np}\PY{o}{.}\PY{n}{arange}\PY{p}{(}\PY{l+m+mi}{3}\PY{p}{)}
\end{Verbatim}

            \begin{Verbatim}[commandchars=\\\{\}]
{\color{outcolor}Out[{\color{outcolor}57}]:} array([[ 1.,  2.,  3.],
                [ 1.,  2.,  3.],
                [ 1.,  2.,  3.]])
\end{Verbatim}
        
    We can also broadcast in two directions at a time:

    \begin{Verbatim}[commandchars=\\\{\}]
{\color{incolor}In [{\color{incolor}43}]:} \PY{n}{a} \PY{o}{=} \PY{n}{np}\PY{o}{.}\PY{n}{arange}\PY{p}{(}\PY{l+m+mi}{3}\PY{p}{)}\PY{o}{.}\PY{n}{reshape}\PY{p}{(}\PY{p}{(}\PY{l+m+mi}{3}\PY{p}{,} \PY{l+m+mi}{1}\PY{p}{)}\PY{p}{)}
         \PY{n}{b} \PY{o}{=} \PY{n}{np}\PY{o}{.}\PY{n}{arange}\PY{p}{(}\PY{l+m+mi}{3}\PY{p}{)}
         \PY{k}{print} \PY{n}{a}\PY{p}{,} \PY{l+s}{\PYZsq{}}\PY{l+s}{+}\PY{l+s}{\PYZsq{}}\PY{p}{,} \PY{n}{b}\PY{p}{,} \PY{l+s}{\PYZsq{}}\PY{l+s}{=}\PY{l+s+se}{\PYZbs{}n}\PY{l+s}{\PYZsq{}}\PY{p}{,} \PY{n}{a} \PY{o}{+} \PY{n}{b}
\end{Verbatim}

    \begin{Verbatim}[commandchars=\\\{\}]
[[0]
 [1]
 [2]] + [0 1 2] =
[[0 1 2]
 [1 2 3]
 [2 3 4]]
    \end{Verbatim}

    \subsubsection{Rules of Broadcasting}

Broadcasting rules can do the following:

\begin{enumerate}
\def\labelenumi{\arabic{enumi}.}
\item
  If the two arrays differ in their number of dimensions, the shape of
  the array with fewer dimensions is \emph{padded} with ones on its
  leading (left) side.
\item
  If the shape of the two arrays does not match in any dimension, the
  array with shape equal to 1 in that dimension is \emph{stretched} to
  match the other shape.
\item
  If in any dimension the sizes disagree and neither is equal to 1, an
  error is raised.
\end{enumerate}

Note that all of this happens without ever actually creating the
stretched arrays in memory! This broadcasting behavior is in practice
enormously powerful, especially because when numpy broadcasts to create
new dimensions or to `stretch' existing ones, it doesn't actually
replicate the data. In the example above the operation is carried
\emph{as if} the 1.5 was a 1-d array with 1.5 in all of its entries, but
no actual array was ever created. This can save lots of memory in cases
when the arrays in question are large and can have significant
performance implications.

    \subsubsection{Broadcasting Examples:}

So when we do

\begin{verbatim}
np.arange(3) + 5
\end{verbatim}

The scalar 5 is

\begin{itemize}
\itemsep1pt\parskip0pt\parsep0pt
\item
  first `promoted' to a 1-dimensional array of length 1
\item
  then, this array is `stretched' to length 3 to match the first array.
\end{itemize}

After these two operations are complete, the addition can proceed as now
both operands are one-dimensional arrays of length 3.

    When we do

\begin{verbatim}
np.ones((3, 3)) + np.arange(3)
\end{verbatim}

The second array is

\begin{itemize}
\itemsep1pt\parskip0pt\parsep0pt
\item
  first `promoted' to a 2-dimensional array of shape (1, 3)
\item
  then `stretched' to length 3 to match the first array
\end{itemize}

    When we do

\begin{verbatim}
np.arange(3).reshape((3, 1)) + np.arange(3)
\end{verbatim}

\begin{itemize}
\itemsep1pt\parskip0pt\parsep0pt
\item
  the second array is `promoted' to a 2-dimensional array of shape (1,
  3)
\item
  the second array is `stretched' to shape (3, 3)
\item
  the first array is `stretched' to shape (3, 3)
\end{itemize}

Then the operation proceeds as if on two 3 $\times$ 3 arrays

    The general rule is: when operating on two arrays, NumPy compares their
shapes element-wise. It starts with the trailing dimensions, and works
its way forward, creating dimensions of length 1 as needed. Two
dimensions are considered compatible when

\begin{itemize}
\itemsep1pt\parskip0pt\parsep0pt
\item
  they are equal to begin with, or
\item
  one of them is 1; in this case numpy will do the `stretching' to make
  them equal.
\end{itemize}

If these conditions are not met, a
\texttt{ValueError: frames are not aligned} exception is thrown,
indicating that the arrays have incompatible shapes. The size of the
resulting array is the maximum size along each dimension of the input
arrays.

    \begin{Verbatim}[commandchars=\\\{\}]
{\color{incolor}In [{\color{incolor}60}]:} \PY{n}{x} \PY{o}{=} \PY{n}{np}\PY{o}{.}\PY{n}{zeros}\PY{p}{(}\PY{p}{(}\PY{l+m+mi}{3}\PY{p}{,} \PY{l+m+mi}{4}\PY{p}{)}\PY{p}{)}
         \PY{n}{y} \PY{o}{=} \PY{n}{np}\PY{o}{.}\PY{n}{zeros}\PY{p}{(}\PY{p}{(}\PY{l+m+mi}{3}\PY{p}{,} \PY{l+m+mi}{1}\PY{p}{)}\PY{p}{)}
         
         \PY{n}{x} \PY{o}{+} \PY{n}{y}
\end{Verbatim}

            \begin{Verbatim}[commandchars=\\\{\}]
{\color{outcolor}Out[{\color{outcolor}60}]:} array([[ 0.,  0.,  0.,  0.],
                [ 0.,  0.,  0.,  0.],
                [ 0.,  0.,  0.,  0.]])
\end{Verbatim}
        
    \subsubsection{Visualizing Broadcasting}

\begin{figure}[htbp]
\centering
\includegraphics{files/images/fig_broadcast_visual_1.png}
\caption{Illustration of broadcasting}
\end{figure}

(\href{http://www.astroml.org/book_figures/appendix/fig_broadcast_visual.html}{image
source})

    \subsubsection{Questions:}

Will the following broadcasting operations work?

    \begin{Verbatim}[commandchars=\\\{\}]
{\color{incolor}In [{\color{incolor}63}]:} \PY{n}{arr1} \PY{o}{=} \PY{n}{np}\PY{o}{.}\PY{n}{ones}\PY{p}{(}\PY{p}{(}\PY{l+m+mi}{2}\PY{p}{,} \PY{l+m+mi}{3}\PY{p}{)}\PY{p}{)}
         \PY{n}{arr2} \PY{o}{=} \PY{n}{np}\PY{o}{.}\PY{n}{ones}\PY{p}{(}\PY{p}{(}\PY{l+m+mi}{2}\PY{p}{,} \PY{l+m+mi}{1}\PY{p}{)}\PY{p}{)}
         
         \PY{n}{arr1} \PY{o}{+} \PY{n}{arr2}
\end{Verbatim}

            \begin{Verbatim}[commandchars=\\\{\}]
{\color{outcolor}Out[{\color{outcolor}63}]:} array([[ 2.,  2.,  2.],
                [ 2.,  2.,  2.]])
\end{Verbatim}
        
    \begin{Verbatim}[commandchars=\\\{\}]
{\color{incolor}In [{\color{incolor}65}]:} \PY{n}{arr1} \PY{o}{=} \PY{n}{np}\PY{o}{.}\PY{n}{ones}\PY{p}{(}\PY{p}{(}\PY{l+m+mi}{2}\PY{p}{,} \PY{l+m+mi}{3}\PY{p}{)}\PY{p}{)}
         \PY{n}{arr2} \PY{o}{=} \PY{n}{np}\PY{o}{.}\PY{n}{ones}\PY{p}{(}\PY{l+m+mi}{3}\PY{p}{)}
         
         \PY{n}{arr1} \PY{o}{+} \PY{n}{arr2}
\end{Verbatim}

            \begin{Verbatim}[commandchars=\\\{\}]
{\color{outcolor}Out[{\color{outcolor}65}]:} array([[ 2.,  2.,  2.],
                [ 2.,  2.,  2.]])
\end{Verbatim}
        
    \begin{Verbatim}[commandchars=\\\{\}]
{\color{incolor}In [{\color{incolor}67}]:} \PY{n}{arr1} \PY{o}{=} \PY{n}{np}\PY{o}{.}\PY{n}{ones}\PY{p}{(}\PY{p}{(}\PY{l+m+mi}{1}\PY{p}{,} \PY{l+m+mi}{3}\PY{p}{)}\PY{p}{)}
         \PY{n}{arr2} \PY{o}{=} \PY{n}{np}\PY{o}{.}\PY{n}{ones}\PY{p}{(}\PY{p}{(}\PY{l+m+mi}{2}\PY{p}{,} \PY{l+m+mi}{1}\PY{p}{)}\PY{p}{)}
         
         \PY{n}{arr1} \PY{o}{+} \PY{n}{arr2}
\end{Verbatim}

            \begin{Verbatim}[commandchars=\\\{\}]
{\color{outcolor}Out[{\color{outcolor}67}]:} array([[ 2.,  2.,  2.],
                [ 2.,  2.,  2.]])
\end{Verbatim}
        
    \subsubsection{Quick Exercise:}

Use \texttt{np.arange} and \texttt{reshape} to create the array

\begin{verbatim}
A = [[1 2 3 4]
     [5 6 7 8]]
\end{verbatim}

Use \texttt{np.arange} to create the array

\begin{verbatim}
B = [1 2]
\end{verbatim}

Use broadcasting to add \texttt{B} to each \textbf{column} of \texttt{A}
to create the final array

\begin{verbatim}
A + B = [[2  3  4  5]
         [7  8  9 10]
\end{verbatim}

Hint: what shape does \texttt{B} have to be changed to?

    \begin{Verbatim}[commandchars=\\\{\}]
{\color{incolor}In [{\color{incolor}75}]:} \PY{n}{A} \PY{o}{=} \PY{n}{np}\PY{o}{.}\PY{n}{arange}\PY{p}{(}\PY{l+m+mi}{1}\PY{p}{,} \PY{l+m+mi}{9}\PY{p}{)}\PY{o}{.}\PY{n}{reshape}\PY{p}{(}\PY{p}{(}\PY{l+m+mi}{2}\PY{p}{,} \PY{l+m+mi}{4}\PY{p}{)}\PY{p}{)}
         \PY{n}{B} \PY{o}{=} \PY{n}{np}\PY{o}{.}\PY{n}{arange}\PY{p}{(}\PY{l+m+mi}{1}\PY{p}{,} \PY{l+m+mi}{3}\PY{p}{)}
         \PY{n}{A} \PY{o}{+} \PY{n}{B}\PY{o}{.}\PY{n}{reshape}\PY{p}{(}\PY{p}{(}\PY{l+m+mi}{2}\PY{p}{,} \PY{l+m+mi}{1}\PY{p}{)}\PY{p}{)}
\end{Verbatim}

            \begin{Verbatim}[commandchars=\\\{\}]
{\color{outcolor}Out[{\color{outcolor}75}]:} array([[ 2,  3,  4,  5],
                [ 7,  8,  9, 10]])
\end{Verbatim}
        
    \subsection{Reshape and newaxis}

    Rather than specifying the length of the array all of the time with the
reshape method, there is a shortand meaning for ``-1''

    \begin{Verbatim}[commandchars=\\\{\}]
{\color{incolor}In [{\color{incolor}3}]:} \PY{n}{np}\PY{o}{.}\PY{n}{arange}\PY{p}{(}\PY{l+m+mi}{1}\PY{p}{,} \PY{l+m+mi}{9}\PY{p}{)}\PY{o}{.}\PY{n}{reshape}\PY{p}{(}\PY{p}{(}\PY{l+m+mi}{1}\PY{p}{,} \PY{o}{\PYZhy{}}\PY{l+m+mi}{1}\PY{p}{)}\PY{p}{)}\PY{o}{.}\PY{n}{shape}
\end{Verbatim}

            \begin{Verbatim}[commandchars=\\\{\}]
{\color{outcolor}Out[{\color{outcolor}3}]:} (1, 8)
\end{Verbatim}
        
    \begin{Verbatim}[commandchars=\\\{\}]
{\color{incolor}In [{\color{incolor}6}]:} \PY{n}{np}\PY{o}{.}\PY{n}{arange}\PY{p}{(}\PY{l+m+mi}{1}\PY{p}{,} \PY{l+m+mi}{9}\PY{p}{)}\PY{o}{.}\PY{n}{reshape}\PY{p}{(}\PY{p}{(}\PY{l+m+mi}{2}\PY{p}{,} \PY{o}{\PYZhy{}}\PY{l+m+mi}{1}\PY{p}{)}\PY{p}{)}\PY{o}{.}\PY{n}{shape}
\end{Verbatim}

            \begin{Verbatim}[commandchars=\\\{\}]
{\color{outcolor}Out[{\color{outcolor}6}]:} (2, 4)
\end{Verbatim}
        
    Another way to change the shape of \texttt{B} is to use the
\texttt{newaxis} keyword:

    \begin{Verbatim}[commandchars=\\\{\}]
{\color{incolor}In [{\color{incolor}79}]:} \PY{k}{print} \PY{n}{B}\PY{o}{.}\PY{n}{shape}
         \PY{k}{print} \PY{n}{B}\PY{p}{[}\PY{n}{np}\PY{o}{.}\PY{n}{newaxis}\PY{p}{,} \PY{p}{:}\PY{p}{,} \PY{n}{np}\PY{o}{.}\PY{n}{newaxis}\PY{p}{,} \PY{n}{np}\PY{o}{.}\PY{n}{newaxis}\PY{p}{]}\PY{o}{.}\PY{n}{shape}
\end{Verbatim}

    \begin{Verbatim}[commandchars=\\\{\}]
(2,)
(1, 2, 1, 1)
    \end{Verbatim}

    \begin{Verbatim}[commandchars=\\\{\}]
{\color{incolor}In [{\color{incolor}82}]:} \PY{n}{B} \PY{o}{=} \PY{n}{np}\PY{o}{.}\PY{n}{arange}\PY{p}{(}\PY{l+m+mi}{4}\PY{p}{)}
         \PY{n}{B}\PY{p}{[}\PY{p}{:}\PY{p}{,} \PY{n}{np}\PY{o}{.}\PY{n}{newaxis}\PY{p}{]}
\end{Verbatim}

            \begin{Verbatim}[commandchars=\\\{\}]
{\color{outcolor}Out[{\color{outcolor}82}]:} array([[0, 1],
                [2, 3]])
\end{Verbatim}
        
    \subsubsection{Element-wise Functions}

As we mentioned before, Numpy ships with a full complement of
mathematical functions that work on entire arrays, including logarithms,
exponentials, trigonometric and hyperbolic trigonometric functions, etc.
Furthermore, scipy ships a rich special function library in the
\texttt{scipy.special} module that includes Bessel, Airy, Fresnel,
Laguerre and other classical special functions. For example, sampling
the sine function at 100 points between $0$ and $2\pi$ is as simple as:

    \begin{Verbatim}[commandchars=\\\{\}]
{\color{incolor}In [{\color{incolor}87}]:} \PY{n}{x} \PY{o}{=} \PY{n}{np}\PY{o}{.}\PY{n}{linspace}\PY{p}{(}\PY{l+m+mi}{0}\PY{p}{,} \PY{l+m+mi}{2}\PY{o}{*}\PY{n}{np}\PY{o}{.}\PY{n}{pi}\PY{p}{,} \PY{l+m+mi}{100}\PY{p}{)}
         \PY{n}{y} \PY{o}{=} \PY{n}{np}\PY{o}{.}\PY{n}{sin}\PY{p}{(}\PY{n}{x}\PY{p}{)}
\end{Verbatim}


    \subsection{Linear algebra in numpy}


    Numpy ships with a basic linear algebra library, and all arrays have a
\texttt{dot} method whose behavior is that of the scalar dot product
when its arguments are vectors (one-dimensional arrays) and the
traditional matrix multiplication when one or both of its arguments are
two-dimensional arrays:

    \begin{Verbatim}[commandchars=\\\{\}]
{\color{incolor}In [{\color{incolor}88}]:} \PY{n}{v1} \PY{o}{=} \PY{n}{np}\PY{o}{.}\PY{n}{array}\PY{p}{(}\PY{p}{[}\PY{l+m+mi}{2}\PY{p}{,} \PY{l+m+mi}{3}\PY{p}{,} \PY{l+m+mi}{4}\PY{p}{]}\PY{p}{)}
         \PY{n}{v2} \PY{o}{=} \PY{n}{np}\PY{o}{.}\PY{n}{array}\PY{p}{(}\PY{p}{[}\PY{l+m+mi}{1}\PY{p}{,} \PY{l+m+mi}{0}\PY{p}{,} \PY{l+m+mi}{1}\PY{p}{]}\PY{p}{)}
         
         \PY{k}{print} \PY{n}{v1}\PY{p}{,} \PY{l+s}{\PYZsq{}}\PY{l+s}{.}\PY{l+s}{\PYZsq{}}\PY{p}{,} \PY{n}{v2}\PY{p}{,} \PY{l+s}{\PYZsq{}}\PY{l+s}{=}\PY{l+s}{\PYZsq{}}\PY{p}{,} \PY{n}{np}\PY{o}{.}\PY{n}{dot}\PY{p}{(}\PY{n}{v1}\PY{p}{,} \PY{n}{v2}\PY{p}{)}
\end{Verbatim}

    \begin{Verbatim}[commandchars=\\\{\}]
[2 3 4] . [1 0 1] = 6
    \end{Verbatim}

    Here is a regular matrix-vector multiplication, note that the array
\texttt{v1} should be viewed as a \emph{column} vector in traditional
linear algebra notation; numpy makes no distinction between row and
column vectors and simply verifies that the dimensions match the
required rules of matrix multiplication, in this case we have a
$2 \times 3$ matrix multiplied by a 3-vector, which produces a 2-vector:

    \begin{Verbatim}[commandchars=\\\{\}]
{\color{incolor}In [{\color{incolor}89}]:} \PY{n}{A} \PY{o}{=} \PY{n}{np}\PY{o}{.}\PY{n}{arange}\PY{p}{(}\PY{l+m+mi}{6}\PY{p}{)}\PY{o}{.}\PY{n}{reshape}\PY{p}{(}\PY{l+m+mi}{2}\PY{p}{,} \PY{l+m+mi}{3}\PY{p}{)}
         \PY{k}{print} \PY{n}{A}\PY{p}{,} \PY{l+s}{\PYZsq{}}\PY{l+s}{x}\PY{l+s}{\PYZsq{}}\PY{p}{,} \PY{n}{v1}\PY{p}{,} \PY{l+s}{\PYZsq{}}\PY{l+s}{=}\PY{l+s}{\PYZsq{}}\PY{p}{,} \PY{n}{np}\PY{o}{.}\PY{n}{dot}\PY{p}{(}\PY{n}{A}\PY{p}{,} \PY{n}{v1}\PY{p}{)}
\end{Verbatim}

    \begin{Verbatim}[commandchars=\\\{\}]
[[0 1 2]
 [3 4 5]] x [2 3 4] = [11 38]
    \end{Verbatim}

    For matrix-matrix multiplication, the same dimension-matching rules must
be satisfied, e.g.~consider the difference between $A \times A^T$:

    \begin{Verbatim}[commandchars=\\\{\}]
{\color{incolor}In [{\color{incolor}90}]:} \PY{k}{print} \PY{n}{np}\PY{o}{.}\PY{n}{dot}\PY{p}{(}\PY{n}{A}\PY{p}{,} \PY{n}{A}\PY{o}{.}\PY{n}{T}\PY{p}{)}
\end{Verbatim}

    \begin{Verbatim}[commandchars=\\\{\}]
[[ 5 14]
 [14 50]]
    \end{Verbatim}

    and $A^T \times A$:

    \begin{Verbatim}[commandchars=\\\{\}]
{\color{incolor}In [{\color{incolor}91}]:} \PY{k}{print} \PY{n}{np}\PY{o}{.}\PY{n}{dot}\PY{p}{(}\PY{n}{A}\PY{o}{.}\PY{n}{T}\PY{p}{,} \PY{n}{A}\PY{p}{)}
\end{Verbatim}

    \begin{Verbatim}[commandchars=\\\{\}]
[[ 9 12 15]
 [12 17 22]
 [15 22 29]]
    \end{Verbatim}

    \subsection{Reading and writing arrays to disk}

Numpy lets you read and write arrays into files in a number of ways. In
order to use these tools well, it is critical to understand the
difference between a \emph{text} and a \emph{binary} file containing
numerical data. In a text file, the number $\pi$ could be written as
``3.141592653589793'', for example: a string of digits that a human can
read, with in this case 15 decimal digits. In contrast, that same number
written to a binary file would be encoded as 8 characters (bytes) that
are not readable by a human but which contain the exact same data that
the variable \texttt{pi} had in the computer's memory.

The tradeoffs between the two modes are thus:

\begin{itemize}
\item
  Text mode: occupies more space, precision can be lost (if not all
  digits are written to disk), but is readable and editable by hand with
  a text editor. Can \emph{only} be used for one- and two-dimensional
  arrays.
\item
  Binary mode: compact and exact representation of the data in memory,
  can't be read or edited by hand. Arrays of any size and dimensionality
  can be saved and read without loss of information.
\end{itemize}

    \paragraph{Text data}

First, let's see how to read and write arrays in text mode. The
\texttt{np.savetxt} function saves an array to a text file, with options
to control the precision, separators and even adding a header:

    \begin{Verbatim}[commandchars=\\\{\}]
{\color{incolor}In [{\color{incolor}19}]:} \PY{n}{arr} \PY{o}{=} \PY{n}{np}\PY{o}{.}\PY{n}{linspace}\PY{p}{(}\PY{l+m+mi}{0}\PY{p}{,} \PY{l+m+mi}{15}\PY{p}{,} \PY{l+m+mi}{10}\PY{p}{)}\PY{o}{.}\PY{n}{reshape}\PY{p}{(}\PY{l+m+mi}{2}\PY{p}{,} \PY{l+m+mi}{5}\PY{p}{)}
         \PY{k}{print} \PY{n}{arr}
         
         \PY{n}{np}\PY{o}{.}\PY{n}{savetxt}\PY{p}{(}\PY{l+s}{\PYZsq{}}\PY{l+s}{test.out}\PY{l+s}{\PYZsq{}}\PY{p}{,} \PY{n}{arr}\PY{p}{,} \PY{n}{fmt}\PY{o}{=}\PY{l+s}{\PYZsq{}}\PY{l+s+si}{\PYZpc{}.2e}\PY{l+s}{\PYZsq{}}\PY{p}{,} \PY{n}{header}\PY{o}{=}\PY{l+s}{\PYZdq{}}\PY{l+s}{My dataset}\PY{l+s}{\PYZdq{}}\PY{p}{)}
         \PY{o}{!}cat test.out
\end{Verbatim}

    \begin{Verbatim}[commandchars=\\\{\}]
[[  0.           1.66666667   3.33333333   5.           6.66666667]
 [  8.33333333  10.          11.66666667  13.33333333  15.        ]]
\# My dataset
0.00e+00 1.67e+00 3.33e+00 5.00e+00 6.67e+00
8.33e+00 1.00e+01 1.17e+01 1.33e+01 1.50e+01
    \end{Verbatim}

    And this same type of file can then be read with the matching
\texttt{np.loadtxt} function:

    \begin{Verbatim}[commandchars=\\\{\}]
{\color{incolor}In [{\color{incolor}18}]:} \PY{n}{arr2} \PY{o}{=} \PY{n}{np}\PY{o}{.}\PY{n}{loadtxt}\PY{p}{(}\PY{l+s}{\PYZsq{}}\PY{l+s}{test.out}\PY{l+s}{\PYZsq{}}\PY{p}{)}
         \PY{k}{print} \PY{n}{arr2}
\end{Verbatim}

    \begin{Verbatim}[commandchars=\\\{\}]
[[  0.     1.67   3.33   5.     6.67]
 [  8.33  10.    11.7   13.3   15.  ]]
    \end{Verbatim}

    \paragraph{Binary Data}

For binary data, Numpy provides the \texttt{np.save} and
\texttt{np.savez} routines. The first saves a single array to a file
with \texttt{.npy} extension, while the latter can be used to save a
\emph{group} of arrays into a single file with \texttt{.npz} extension.
The files created with these routines can then be read with the
\texttt{np.load} function.

Let us first see how to use the simpler \texttt{np.save} function to
save a single array:

    \begin{Verbatim}[commandchars=\\\{\}]
{\color{incolor}In [{\color{incolor}20}]:} \PY{n}{np}\PY{o}{.}\PY{n}{save}\PY{p}{(}\PY{l+s}{\PYZsq{}}\PY{l+s}{test.npy}\PY{l+s}{\PYZsq{}}\PY{p}{,} \PY{n}{arr}\PY{p}{)}
         
         \PY{c}{\PYZsh{} Now we read this back}
         \PY{n}{arr2n} \PY{o}{=} \PY{n}{np}\PY{o}{.}\PY{n}{load}\PY{p}{(}\PY{l+s}{\PYZsq{}}\PY{l+s}{test.npy}\PY{l+s}{\PYZsq{}}\PY{p}{)}
         \PY{k}{print} \PY{n}{arr2n}
\end{Verbatim}

    \begin{Verbatim}[commandchars=\\\{\}]
[[  0.           1.66666667   3.33333333   5.           6.66666667]
 [  8.33333333  10.          11.66666667  13.33333333  15.        ]]
    \end{Verbatim}

    Now let us see how the \texttt{np.savez} function works. You give it a
filename and either a sequence of arrays or a set of keywords. In the
first mode, the function will auotmatically name the saved arrays in the
archive as \texttt{arr\_0}, \texttt{arr\_1}, etc:

    \begin{Verbatim}[commandchars=\\\{\}]
{\color{incolor}In [{\color{incolor}21}]:} \PY{n}{np}\PY{o}{.}\PY{n}{savez}\PY{p}{(}\PY{l+s}{\PYZsq{}}\PY{l+s}{test.npz}\PY{l+s}{\PYZsq{}}\PY{p}{,} \PY{n}{arr}\PY{p}{,} \PY{n}{arr2}\PY{p}{)}
         \PY{n}{arrays} \PY{o}{=} \PY{n}{np}\PY{o}{.}\PY{n}{load}\PY{p}{(}\PY{l+s}{\PYZsq{}}\PY{l+s}{test.npz}\PY{l+s}{\PYZsq{}}\PY{p}{)}
         \PY{n}{arrays}\PY{o}{.}\PY{n}{files}
         
         \PY{n}{arrays}\PY{p}{[}\PY{l+s}{\PYZsq{}}\PY{l+s}{arr\PYZus{}1}\PY{l+s}{\PYZsq{}}\PY{p}{]}
\end{Verbatim}

            \begin{Verbatim}[commandchars=\\\{\}]
{\color{outcolor}Out[{\color{outcolor}21}]:} array([[  0.  ,   1.67,   3.33,   5.  ,   6.67],
                [  8.33,  10.  ,  11.7 ,  13.3 ,  15.  ]])
\end{Verbatim}
        
    \subsubsection{Loading from more complex formats}

TODO

    \section{Breakout: trapezoidal integration}

\textbf{Illustrates}: basic array slicing, functions as first class
objects.

In this exercise, you are tasked with implementing the simple trapezoid
rule formula for numerical integration. If we want to compute the
definite integral

\[
     \int_{a}^{b}f(x)dx
\]

we can partition the integration interval $[a,b]$ into smaller
subintervals, and approximate the area under the curve for each
subinterval by the area of the trapezoid created by linearly
interpolating between the two function values at each end of the
subinterval:

\begin{figure}[htbp]
\centering
\includegraphics{files/images/trapezoidal_rule.png}
\caption{Illustration of the trapezoidal rule}
\end{figure}

The blue line represents the function $f(x)$ and the red line is the
linear interpolation. By subdividing the interval $[a,b]$, the area
under $f(x)$ can thus be approximated as the sum of the areas of all the
resulting trapezoids.

If we denote by $x_{i}$ ($i=0,\ldots,n,$ with $x_{0}=a$ and $x_{n}=b$)
the abscissas where the function is sampled, then

\[
   \int_{a}^{b}f(x)dx\approx\frac{1}{2}\sum_{i=1}^{n}\left(x_{i}-x_{i-1}\right)\left(f(x_{i})+f(x_{i-1})\right).
\]

The common case of using equally spaced abscissas with spacing
$h=(b-a)/n$ reads simply

\[
   \int_{a}^{b}f(x)dx\approx\frac{h}{2}\sum_{i=1}^{n}\left(f(x_{i})+f(x_{i-1})\right).
\]

One frequently receives the function values already precomputed,
$y_{i}=f(x_{i}),$ so the equation above becomes

\[
   \int_{a}^{b}f(x)dx\approx\frac{1}{2}\sum_{i=1}^{n}\left(x_{i}-x_{i-1}\right)\left(y_{i}+y_{i-1}\right).
\]

    TODO: Add naive Python implementation with loops, to simplify the task.

    \subsubsection{Exercises}

\paragraph{Part 1}

Write a function \texttt{trapz(x, y)}, that applies the trapezoid
formula to pre-computed values, where \texttt{x} and \texttt{y} are 1-d
arrays. (Use numpy operations, not loops!)

    \begin{Verbatim}[commandchars=\\\{\}]
{\color{incolor}In [{\color{incolor}}]:} 
\end{Verbatim}

    \paragraph{Part 2}

Write a function \texttt{trapzf(f, a, b, npts=100)} that accepts a
function \texttt{f}, the endpoints \texttt{a} and \texttt{b} and the
number of samples to take \texttt{npts}. Sample the function uniformly
at these points and return the value of the integral.

    \begin{Verbatim}[commandchars=\\\{\}]
{\color{incolor}In [{\color{incolor}}]:} 
\end{Verbatim}

    \paragraph{Part 3}

Verify that both functions above are correct by showing that they
produce correct values for a simple integral such as $\int_0^3 x^2$.

    \begin{Verbatim}[commandchars=\\\{\}]
{\color{incolor}In [{\color{incolor}}]:} 
\end{Verbatim}

    \paragraph{Part 4}

Preview of what's to come: use the documentation features of IPython to
explore the submodule \texttt{scipy.integrage}. Can you find a suitable
function to perform this integration above?

    \begin{Verbatim}[commandchars=\\\{\}]
{\color{incolor}In [{\color{incolor}}]:} 
\end{Verbatim}


    % Add a bibliography block to the postdoc
    
    
    
    \end{document}
